\chapter*{\practicetitle}
\textbf{Цель работы:} Ознакомиться с архитектурой SDR-устройства ADALM Pluto. Сформировать радиоприемник для приёма и воспроизведения радиосигналов в реальном времени при помощи фреймворка GNU Radio и Adalm Pluto SDR.
\addcontentsline{toc}{section}{\MakeUppercase{Цель работы}}

\sect{Краткие теоретические сведения}
\textbf{GNURadio} - это инструмент который позволяет при помощи “строительных блоков” создать конфигурацию радиоустройства, не написав ни одной строчки кода, и запустить программу непосредственно с использованием модуля \textbf{SDR} (программно-определяемое радио), например Adalm-Pluto, LimeSDR, и др.

{\setlength{\parskip}{8pt}\textbf{Что такое SDR?} \\
\textbf{SDR} (Software Defined Radio) — это программно-определяемое радио, в котором большинство функций традиционного радиоприёмника и радиопередатчика реализуются программно, а не аппаратно.\\В классических радиосистемах такие операции, как фильтрация, модуляция, демодуляция, обработка спектра и синхронизация выполняются с помощью аналоговых электронных блоков: фильтров, смесителей, детекторов, генераторов и т. д.\\В SDR эти операции переносятся в цифровую область и обрабатываются программно, на компьютере или встроенном процессоре. Аппаратная часть SDR сведена к минимуму и включает только аналоговый фронтенд, АЦП и ЦАП, необходимые для преобразования сигнала между аналоговой и цифровой формами. Благодаря этому SDR позволяет быстро изменять параметры работы радиосистемы — частоту, полосу, тип модуляции и другие настройки — без переделки оборудования, лишь за счёт изменения программной конфигурации.

\textbf{Adalm Pluto} имеет два основных компонента — аналоговый радиочастотный приемопередатчик \textbf{AD9363} и система на кристалле (SoC) Xilinx Zynq 7000 Series.

\textbf{AD9363} содержит необходимые усилители, фильтры, а также цифро-аналоговые и аналого-цифровые преобразователи (\textbf{12-bit ADC} и \textbf{}{DAC}). Позволяет принимать и предоставлять IQ-сэмплы, формируемые  или принимаемые системой на кристале \textbf{Zynq SoC}. Пользователь может настроить  несущую частоту, частоту дискретизации и т.д.

Система на кристале оснащена процессором \textbf{ARM Cortex A9}, работающим на частоте \textbf{667 МГц} в сочетании с программируемой пользователем вентильной матрицей (\textbf{FPGA}). Пользовательские аппаратные модули реализованы в FPGA, обеспечивая связующий слой между прикладными процессорами (\textbf{ARM Cortex}) и радиочастотным приемопередатчиком (\textbf{AD9363}). Позволяет передавать и получать IQ-сэмплы пользовательскими приложениями, работающими на прикладных процессорах, через драйверы, размещенные в ядре \textbf{Linux}.

\begin{figure}[H]
  \centering
  \includegraphics[height=0.5\textwidth, keepaspectratio]{images/\practicenumber/adalmplutodiagram}
  \caption{Структура SDR}
\end{figure}

\begin{figure}[H]
  \centering
  \includegraphics[height=0.5\textwidth, keepaspectratio]{images/\practicenumber/adalmpluto}
  \caption{Устройство SDR Adalm Pluto}
\end{figure}}

\newpage
\sect{Ход работы}
\textbf{1. Установка GNURadio}\\
В большинстве версий Ubuntu в менеджере приложений (Ubuntu Software) присутствует пакет GNU Radio Companion.\\
Я установил GNU Radio через менеджер приложений - Ubuntu Software Center.

\textbf{2. Сборка FM-радио приёмника}\\
\textbf{Необходимые блоки:}
\begin{itemize}
  \item {\textbf{PlutoSDR Source}} - источник сигнала с устройства ADALM Pluto.
        \begin{itemize}
          \item IIO context URI - IP адрес для подключения к устройству.
          \item LO Frequency - несущая частота FM-станции.
          \item Sample Rate - частота дискретизации.
          \item Buffer Size - размер буфера.
          \item RF Bandwidth - ширина полосы пропускания.
        \end{itemize}
        \begin{figure}[H]
          \centering
          \includegraphics[width=0.5\textwidth, keepaspectratio]{images/\practicenumber/PlutoSDR_source}
          \caption{PlutoSDR Source}
        \end{figure}
  \item {\textbf{QT GUI Frequency Sink}} - блок для визуализации спектра сигнала в реальном времени.\\Поможет нам визуально для более точного поиска FM-частоты.
        \begin{itemize}
          \item FFT Size - размер БПФ.
          \item Center Frequency - центральная частота.
          \item Bandwidth - полоса пропускания.
        \end{itemize}
        \begin{figure}[H]
          \centering
          \includegraphics[width=0.5\textwidth, keepaspectratio]{images/\practicenumber/Freq_sink}
          \caption{QT GUI Frequency Sink}
        \end{figure}
  \item {\textbf{Low Pass Filter}} - блок для фильтрации шумов.\\Позволяет избавиться (подавить) от “лишнего” сигнала на частотах, отличных от среза искомой полосы FM-станции.
        \begin{itemize}
          \item Decimation - параметр, который необходимо настроить под sample rate аудио-потока для прослушивания.
          \item Gain - коэффициент усиления.
          \item Sample Rate - частота дискретизации.
          \item Cutoff Frequency - частота среза фильтра.
          \item Transition Width - ширина переходной полосы.
          \item Window - тип окна.
        \end{itemize}
        \begin{figure}[H]
          \centering
          \includegraphics[width=0.5\textwidth, keepaspectratio]{images/\practicenumber/LowPassFilter}
          \caption{Low Pass Filter}
        \end{figure}
  \item {\textbf{WBFM Receive}} - блок позволяющий демодулировать широковещательный FM-сигнал.
        \begin{itemize}
          \item Quadrature Rate - частота дискретизации.
          \item Audio Decimation - параметр, который необходимо настроить под sample rate аудио-потока для прослушивания.
        \end{itemize}
        \begin{figure}[H]
          \centering
          \includegraphics[width=0.5\textwidth, keepaspectratio]{images/\practicenumber/wbfm}
          \caption{WBFM Receive}
        \end{figure}
  \item {\textbf{Audio Sink}} - блок для вывода звука на аудиоустройство
        \begin{itemize}
          \item Sample Rate - частота дискретизации.
        \end{itemize}
        \begin{figure}[H]
          \centering
          \includegraphics[width=0.5\textwidth, keepaspectratio]{images/\practicenumber/audioSink}
          \caption{Audio Sink}
        \end{figure}
  \item {\textbf{QT GUI Time Sink}} - блок для визуализации сигнала во временной области.
        \begin{itemize}
          \item Number of Points - количество точек на экране.
          \item Sample Rate - частота дискретизации.
          \item Autoscale - авто масштабирование.
        \end{itemize}
        \begin{figure}[H]
          \centering
          \includegraphics[width=0.5\textwidth, keepaspectratio]{images/\practicenumber/Time_sink}
          \caption{QT GUI Time Sink}
        \end{figure}
  \item {\textbf{Variables}} - блок переменных.
        \begin{itemize}
          \item samp\_rate: 2.4M - частота дискретизации.
          \item bw: 480k - ширина полосы пропускания.
        \end{itemize}
  \item {\textbf{QT GUI Range}} - GUI ползунок для изменения каких либо параметров в реальном времени.
        \begin{itemize}
          \item ID: tune - идентификатор переменной.
          \item Default Value: 106.2M - значение по умолчанию.
          \item Start: 88.7M - минимальное значение.
          \item Stop: 110.1M - максимальное значение.
          \item Step: 12.5k - шаг изменения значения.
        \end{itemize}
\end{itemize}

\textbf{Итоговая схема FM-радио приёмника:}
\begin{figure}[H]
  \centering
  \includegraphics[width=1\textwidth, keepaspectratio]{images/\practicenumber/scheme}
  \caption{FM Radio Receiver}
\end{figure}
\begin{figure}[H]
  \centering
  \includegraphics[width=1\textwidth, keepaspectratio]{images/\practicenumber/channel}
  \caption{Пример GUI FM-радио приёмника}
\end{figure}

\sect{Вывод}
В ходе выполнения лабораторной работы была изучена архитектура программно-определяемого радио ADALM-Pluto, а также особенности его взаимодействия с фреймворком GNU Radio. На практике был собран и протестирован FM-радиоприёмник, работающий в реальном времени. В процессе работы были освоены базовые блоки GNU Radio, настроены параметры источника сигнала, фильтрации, демодуляции и аудиовывода. Построенная схема позволила успешно принимать, демодулировать и воспроизводить FM-радиосигналы, а также визуализировать спектр и временную область сигнала. Работа показала удобство и гибкость SDR-подхода, позволяющего изменять функциональность радиосистемы за счёт программной конфигурации без необходимости модификации аппаратной части.