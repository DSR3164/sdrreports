\documentclass[a4paper,12pt,oneside,openany]{memoir}
\input{preamble.tex}

\newcommand{\includelab}[2]{% Макрос для включения лабораторной работы
  \renewcommand{\practicenumber}{#1}%
  \renewcommand{\practicetitle}{#2}%
  \phantomsection
  \addcontentsline{toc}{section}{\MakeUppercase{\practicetitle}}%
  \let\oldaddcontentsline\addcontentsline%
  \renewcommand{\addcontentsline}[3]{}%
  \include{\practicenumber}%
  \let\addcontentsline\oldaddcontentsline%
}

\newcommand{\practicenumber}{}
\newcommand{\practicetitle}{}

\begin{document}

\includepdf[pages=-]{ШКЛЯЕВ_ПРАКТИКА_ТИТУЛ_SDR}

\newpage
\setcounter{page}{2}
\OnehalfSpacing*
\tableofcontents*
\newpage

\includelab{1}{Архитектура Adalm Pluto SDR. GNU Radio. Построение радио-приёмника}
\includelab{2}{Введение в архитектуру SDR-устройств. Знакомство с библиотеками Soapy SDR, Libiio. Инициализация SDR-устройства. Работа с буфером: получение I/Q-отсчётов}
\includelab{3}{Принципы работы Soapy SDR и работа с Adalm Pluto. Формирование и передача сигналов произвольной формы}
%\includelab{4}{Архитектура SDR-устройств. Продолжение. Примеры формирования I/Q-сэмплов. Работа с буфером приёма}
%\includelab{5}{Приём сигналов с BPSK/QPSK. Имитация аналоговой передачи звука и его приём на SDR. Влияние чувствительности и усиления}
%\includelab{6}{Моделирование формирования и приёма QPSK. Реализация приёма и передачи BPSK}
%\includelab{7}{Алгоритм дискретной свёртки. Реализация приёма и передачи BPSK-символов}
%\includelab{8}{Приём QPSK/BPSK. Согласованный фильтр, глазковая диаграмма, поиск оптимального отсчёта, необходимость символьной синхронизации}
%\includelab{9}{Символьная синхронизация. Детектор временной ошибки. Схема Гарднера. Реализация детектора на SDR. Написание функций петли (контура) синхронизации}

\end{document}
