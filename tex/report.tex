\documentclass[a4paper,12pt,oneside,openany]{memoir}
\input{preamble.tex}

\newcommand{\includelab}[2]{% Макрос для включения лабораторной работы
  \renewcommand{\practicenumber}{#1}%
  \renewcommand{\practicetitle}{#2}%
  \phantomsection
  \addcontentsline{toc}{section}{\MakeUppercase{\practicetitle}}%
  \let\oldaddcontentsline\addcontentsline%
  \renewcommand{\addcontentsline}[3]{}%
  \include{\practicenumber}%
  \let\addcontentsline\oldaddcontentsline%
}

\newcommand{\practicenumber}{}
\newcommand{\practicetitle}{}

\begin{document}

\includepdf[pages=-]{ШКЛЯЕВ_ПРАКТИКА_ТИТУЛ_SDR}

\newpage
\setcounter{page}{2}
\OnehalfSpacing*
\tableofcontents*
\newpage
\let\oldps@plain\ps@plain
\includelab{1}{Архитектура Adalm Pluto SDR. GNU Radio. Построение радио-приёмника} %16.09
\includelab{2}{Введение в архитектуру SDR-устройств. Знакомство с библиотеками Soapy SDR, Libiio. Инициализация SDR-устройства. Работа с буфером: получение I/Q-отсчётов} % 23.09
\includelab{3}{Принципы работы Soapy SDR и работа с Adalm Pluto. Формирование и передача сигналов произвольной формы} % 30.09, 07.10
\includelab{4}{Имитация аналоговой передачи звука и его приём на SDR. Влияние чувствительности и усиления} % 14.10, 21.10
\includelab{5}{Моделирование формирования и приёма QPSK. Реализация приёма и передачи BPSK. Алгоритм дискретной свёртки.} % 28.10, 11.11
\includelab{6}{Приём QPSK/BPSK. Согласованный фильтр, глазковая диаграмма, поиск оптимального отсчёта, необходимость символьной синхронизации} % 18.11
\includelab{7}{Символьная синхронизация. Детектор временной ошибки. Схема Гарднера. Реализация детектора на SDR. Написание функций петли (контура) синхронизации} % 25.11, 02.12

\end{document}
