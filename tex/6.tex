\let\ps@plain\oldps@plain

\chapter*{\practicetitle} %Приём QPSK/BPSK. Согласованный фильтр, глазковая диаграмма, поиск оптимального отсчёта, необходимость символьной синхронизации
\textbf{Цель работы:} Реализовать согласованный фильтр 
\addcontentsline{toc}{section}{\MakeUppercase{Цель работы}}

\sect{Краткие теоретические сведения}
\textbf{Matched Filter} - фильтр смыслом которого является увеличение отношения сигнал/шум (SNR). Этот эффект достигается методом свертки входного сигнала с импульсной характеристикой формирующего фильтра передатчика, обращенной во времени.
Это позволяет максимизировать выходной сигнал в момент времени принятия решения, что улучшает вероятность правильного принятия символа.

\begin{figure}[H]
  \centering
  \includegraphics[height=1\textwidth, keepaspectratio]{images/\practicenumber/mf}
  \caption{Выход с Matched Filter}
\end{figure}

\begin{figure}[H]
  \centering
  \includegraphics[height=1\textwidth, keepaspectratio]{images/\practicenumber/rx}
  \caption{Упрощенная архитектура приемника (демодулятора)}
\end{figure}

\textbf{Глазковая диаграмма} (Eye Diagram) — это графическое отображение цифрового сигнала, построенное путём наложения нескольких последовательных периодов символа на один интервал времени символа $T_s$. Она предназначена для анализа качества передачи и выявления интерсимвольной интерференции (ISI), шумов и амплитудных искажений.

\sect{Ход работы}

Реализация MF на Python:
\begin{mdframed}
\begin{minted}{python}
rx = np.fromfile(f"/home/plutoSDR/sdr/pluto/dev/rx1.pcm", dtype=np.int16)
samples_rx = []

for x in range(0, len(rx), 2):
    samples_rx.append((rx[x]+ 1j * rx[x+1])/np.max(rx))

a = np.ones(10)
Is = np.real(samples_rx); Qs = np.imag(samples_rx)
pila = np.convolve(Is, a) + 1j*np.convolve(Qs, a)

signal = []
for x in range(9, len(pila), 10): # Оптимальный сдвиг найден экспериментально
    signal.append(pila[x])
\end{minted}
\end{mdframed}

Полученный сигнал после MF:

\begin{figure}[H]
  \centering
  \includegraphics[height=1\textwidth, keepaspectratio]{images/\practicenumber/after_mf}
  \caption{График сигнала после MF, сигнальное созвездие}
\end{figure}

Сдвиг для оптимального отсчёта был найден экспериментально, равен 9. Для нахождения оптимального отсчёта в реальном времени как раз и необходимо использовать автоматическую символьную синхронизацию.

Так же была построена глазковая диаграмма путем взятия отрезков сигнала длиной в $T_s$ то есть 10 отсчётов (длина символа) с шагом в 10 отсчётов и наложением их друг на друга:
\begin{figure}[H]
  \centering
  \includegraphics[height=0.45\textwidth, keepaspectratio]{images/\practicenumber/eye_diagram}
  \caption{Глазковая диаграмма переходов символа между состояниями. (Синий - I, Красный - Q)}
\end{figure}

\sect{Вывод}
Была реализована и протестирована на практике работа согласованного фильтра для приёма QPSK/BPSK сигналов. Проведён анализ полученного сигнала с помощью глазковой диаграммы, что позволило визуально оценить качество передачи и выявить возможные искажения. Работа показала важность использования согласованных фильтров в цифровой связи для улучшения качества приёма сигналов. Так же стало понятно, что для корректного приёма необходимо реализовывать автоматическую символьную синхронизацию для поиска оптимального отсчёта в реальном времени.