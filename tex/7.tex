\chapter*{\practicetitle} %Символьная синхронизация. Детектор временной ошибки. Схема Гарднера. Реализация детектора на SDR. Написание функций петли (контура) синхронизации
\textbf{Цель работы:} Реализовать схему Гарднера, контур синхронизации
\addcontentsline{toc}{section}{\MakeUppercase{Цель работы}}

\sect{Краткие теоретические сведения}

\textbf{TED} - (Timing Error Detection) функциональный узел, оценивающий отклонение момента выборки от оптимального положения внутри символного интервала. TED принимает  сигнал после matched filter и формирует скалярную ошибку e[n], которая затем используется петлёй синхронизации (loop filter) для корректировки фазового накопителя или интерполятора.

\begin{figure}[H]
    \centering
    \includegraphics[width=0.5\textwidth, keepaspectratio]{images/\practicenumber/ted}
    \caption{Визуализация ошибки пика символа из за дискретизации}
\end{figure}

\begin{figure}[H]
    \centering
    \includegraphics[width=1\textwidth, keepaspectratio]{images/\practicenumber/norm_ted}
    \caption{График характеристики TED}
\end{figure}

\textbf{Zero-Crossing} - простейший детектор, построенный на предположении, что фильтрованный манипулированный сигнал симметричен относительно нуля, а символные выборки должны находиться вблизи амплитудных экстремумов. Граница между символами часто проходит в точке изменения знака производной или самого сигнала. Если выборка приходит "не вовремя", то соседние отсчёты будут иметь характерную смену знака, что позволяет оценить временную ошибку.

$e[n]=x[n-\frac{1}{2}](x[n]-x[n-1])$

\begin{figure}[H]
    \centering
    \includegraphics[width=0.5\textwidth, keepaspectratio]{images/\practicenumber/zero-cross}
    \caption{Визуализация работы Zero-Crossing}
\end{figure}

\textbf{Схема Гарднера} - наиболее распространённый TED в QPSK/π/4-QPSK трактах. Он не требует знания фазы несущей (некогерентный), устойчив к шумам и не использует дифференциальные операции, что снижает чувствительность к ISI. Он основан на предположении, что оптимальная точка выборки должна находиться в центре символьного интервала, а выборка в середине промежутка между ними должна быть равна среднему значению соседних символов.

\textbf{Loop Filter} - (фильтр контура синхронизации) элемент петли восстановления времени, который формирует динамику реакции системы на ошибку. Определяет, насколько петля будет агрессивно корректировать фазу выборки, вычисляет сдвиг.

\begin{figure}[H]
    \centering
    \includegraphics[width=1\textwidth, keepaspectratio]{images/\practicenumber/loop-filter}
    \caption{Общая схема с контуром синхронизации}
\end{figure}

Для алгоритма Loop Filter используются коэффициенты:

$\displaystyle \theta=\frac{\frac{B_nT_s}{N_{sps}}}{\zeta + \frac{1}{4\zeta}}$

$\displaystyle K_1=\frac{-4\zeta\theta}{(1+2\zeta\theta+\theta^2)K_p}$

$\displaystyle K_2=\frac{-4\theta^2}{(1+2\zeta\theta+\theta^2)K_p}$
\begin{itemize}
    \item $N_{sps}$ - количество отчетов на символ - $\tau$.
    \item $\zeta$ - коэффициент демпфирования контура синхронизации.
    \item $B_nT_s$ - нормированная полоса пропускания контура $B_n$ на период символа - $T_s$.
    \item $K_p$ - коэффициент пропорциональной составляющей фильтра
\end{itemize}

\newpage
\sect{Вывод}
В ходе работы была изучена схема символьной синхронизации на основе детектора временной ошибки Гарднера. Проведённый анализ показал, что Gardner TED эффективно оценивает сдвиг выборки относительно идеальной точки символа, формируя скалярную ошибку, которая может быть напрямую использована в контуре синхронизации.